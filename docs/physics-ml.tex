\documentclass[12pt]{article} 
\begin{document}

\section{Physics Parallels}

Predictive power is a key measure of successful physical theory. Quantum physics predictions are formulated in terms of
probabilities. This principal limitation comes from existence of incompatible observables
(measurable quantities) i.e. it is not possible to measure the values of these quantities simultaneously.

Models of physical phenomena are mathematical theories. These models are build for centuries according to
(existing) observations and experiments on one side and proven mathematical theories on the other side.
Predictions from these models are verified by new experiments. As a general theories, physical models have its parameters.
Some of them are fixed by already known data (e.g. masses of particles). Unknown parameters provide space for
alternative outcomes in, so far, not experimentally covered area.

Above mentioned schema somehow correlates with probabilistic predictions, parameter tuning to fit the training data
vs. verifying predictions for real data.

We can draw a parallel of mathematical formalism of quantum theory and math used in machine learning.

\end{document}

