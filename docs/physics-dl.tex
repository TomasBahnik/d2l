\documentclass[12pt]{article}
\usepackage{amsmath}
\usepackage{amsfonts}
\usepackage{bm}
\usepackage{physics}
\usepackage[total={7in,10in}]{geometry}
\geometry{a4paper}
\usepackage{hyperref} % Colors for links, text and headings
% used from https://github.com/goodfeli/dlbook_notation
\input{math_commands.tex}

\newcommand{\be}{\begin{equation}}
\newcommand{\ee}{\end{equation}}
\newcommand{\normp}[2]{\norm{\vb*{#1}}_#2}

\begin{document}
% Spacing between notation sections
\newlength{\notationgap}
\setlength{\notationgap}{1pc}

\section{Physics Parallels}

Quantum physics predictions are formulated in terms of
probabilities. This principal limitation comes from existence of incompatible observables
(measurable quantities) when it is not possible to measure the values of these quantities simultaneously.

Energy of the system plays crucial role in quantum and statistical physics.
\begin{itemize}
  \item Lowest-energy state is the ground state
  \item Energy eigenstates (states with defined energy) are stationary states (measurement of observables is time independent)
  \item For some system, requirement of mathematical consistency, limits allowed values of energy eigenvalues i.e energy levels are discrete
\end{itemize}

Quantum physics generally deals with systems at subatomic level. Statistical physics derives its fundamental laws of macroscopic system, combining subatomic principles and general arguments translating extreme complexity of external interaction to the language of probability and statistics. In case of (quantum) statistical physics the inherent probabilistic nature of subatomic world is combined with probabilistic approach to, otherwise hopeless to solve, behavior of macroscopic bodies. 

Important case is statistical equilibrium when the (macroscopic) system is in quantum stationary state. After many heuristics arguments (to be specified) we conclude

\be \log w_n = \alpha + \beta E_n \ee 

where $w_n$ is quantum probability of macroscopic system being in stationary state with energy $E_n$. The entropy can be expressed as average value of $\log w_n$ (to be explained)

\be S = -\sum_n w_n \log w_n \ee

\section{Definitions}
Some definitions from \href{http://www.deeplearningbook.org/}{Deep Learning Book} using their \href{https://github.com/goodfeli/dlbook_notation}{\LaTeX templates}. \href{http://mirrors.ibiblio.org/CTAN/macros/latex/contrib/physics/physics.pdf}{Physics package} provides the same symbols in more generic way eg. random vector $\vb*{x}=$\verb|\vb*{x}| and vector $\vb{x}=$\verb|\vb{x}| instead of separate definition for vector $\vx=$\verb|\vx| and random vector $\rvx=$\verb|\rvx|. Where possible I use symbols from \verb|physics| package like $\normp{x}{p}$ (\verb|\normp{x}{p}|) for $\normlp$


\noindent$\normlp$ norm of $\vx$
\be || \vx ||_p = \left(\sum_i |x_i|^p\right)^\frac{1}{p}\label{eq_normlp}\ee
for $p \in \sR, p \ge 1$. $\normltwo$ is abbreviated as $||\vx|| = \norm{\vb*{x}}$. {\bf Squared} $\normltwo$ norm meaning scalar product $\vx\cdot\vx = \vx^T\vx =\bra{\vx}\ket{\vx}$ (Dirac notation) doesn't have any special symbol.
\be \normmax = \max_i|x_i| \ee  

\section{MXNET functions}

mxnet.gluon.loss.L2Loss : $L = \frac{1}{2} \sum_i \vert {label}_i - {pred}_i \vert^2$. While original definition was

\[\|\mathbf{x}\|_2 = \sqrt{\sum_{i=1}^n x_i^2}\]

\input{notation.tex}

\end{document}

